\documentclass{article}
\usepackage[utf8]{inputenc}
\usepackage{amsmath}
\usepackage{amssymb}
\usepackage{geometry}
\usepackage{hyperref}
\usepackage{xcolor}


\newcommand{\bluelink}[2]{{\color{blue}\underline{\href{#1}{#2}}}}

\geometry{a4paper, left=30mm, right=30mm, top=30mm}

\title{Working of pi-calculation-simulator}
\author{Divyajeet Singh}
\date{August 28, 2022}

\begin{document}
\maketitle

\section*{Preface}
This document was written after creation of the repository
\bluelink{https://github.com/divyajeettt/pi-calculation-simulator}{pi-calculation-simulator},
only to provide a better experience to readers wishing
to go through the proof of why the $\pi$-calculation algorithm is correct.

\section*{Set-Up}

A square $S$ of side $s$ appears on the screen. A circle $C$ of diameter $d$ is inscribed in $S$.
This ensures that side $s$ of square $S$ is exactly equal to diameter $d$ of circle $C$, i.e.

\begin{equation}\label{equality} s = d\end{equation}

\noindent Pairs $(x, y)$ are drawn randomly from a uniform distribution to place darts at.

\section*{Explanation}

The following section contains the explanation of why the algorithm works. \\

Let $N_t$ be the total number of darts to be thrown at $S$.

Let $N_0$ be the number of darts that land inside $C$.

We find that the ratio

\begin{equation}
    \frac{N_0}{N_t} \varpropto \frac{\text{Area}(C)}{\text{Area}(S)}
\end{equation}

Note that (using (\ref{equality})),

\begin{equation}\begin{split}
    \label{area}
    \frac{\text{Area}(C)}{\text{Area}(S)} &= \frac{\pi d^2}{4} \cdot \frac{1}{s^2} \\
    &= \frac{\pi d^2}{4} \cdot \frac{1}{d^2}\ = \frac{\pi}{4}
\end{split}\end{equation}

Note how for larger $N_t$, the following holds true, on account of covering larger area (using (\ref{area})):

\begin{equation}
    \label{limit}
    \lim_{N_t \to \infty} \left( \frac{N_0}{N_t} \right) = \frac{\text{Area}(C)}{\text{Area}(S)} = \frac{\pi}{4}
\end{equation}

This is the reason that $N_t > 1,000$ is suggested by the application. On rearranging (\ref{limit}),

\begin{equation}
    \label{result}
    \lim_{N_t \to \infty} 4 \cdot \left( \frac{N_0}{N_t} \right) = \pi
\end{equation}

Hence the aforementioned (result (\ref{result}))  is a close (convergent) approximation of $\pi$ for large $N_t$.

\section*{Footnotes}
This algorithm cannot be guaranteed to work. This can be attributed to the fact that each location $(x,y)$ is
chosen randomly, where $x$ and $y$ are chosen randomly from uniform distributions. In the event (of minuscule
probability) that most/all points lie outside/inside $C$, the algorithm is bound to fail.

\noindent Hence, it must only be looked at as an approximation.

\section*{References}
\begin{enumerate}
    \item \bluelink{https://www.youtube.com/watch?v=M34TO71SKGk}{Calculating Pi with Darts (YouTube)}
    \item \bluelink{https://www.cs.wustl.edu/~cytron/cs101/Lectures/5.html}{Computing PI by throwing darts}
\end{enumerate}

\end{document}
